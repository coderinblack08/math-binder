\documentclass[12pt, a4paper]{article}
\usepackage[utf8]{inputenc}
\usepackage[english]{babel}
\renewcommand{\baselinestretch}{1.6}
\setlength{\parskip}{1em}
\newtheorem{theorem}{Theorem}
\newtheorem{formula}{Formula}

\title{\textbf{Math Binder - AMC/AIME}}
\author{Hansel Grimes}

\begin{document}
  \maketitle
  \tableofcontents

  \newpage

  \section{Polynomials}
  \subsection{Definition and Factorization}
  A polynomial is defined as $P(x) = a_nx^n + a_{n-1}x^{n-1} + \cdots + a_2x^2 + a_1x + a_0$ 
  with names corresponding to their degree (constant, linear, quadratic, cubic, quartic).
  
  The factored form is written as $P(x) = a(x-r)(x-p)\cdots(x-q)$.
  The simplest and most useful polynomial is the quadratic. It can be written as $ax^2+bx+c$ and factored respectively.
  The formula to solve for $x$ is $x=\frac{-b \pm \sqrt{b^2 - 4ac}}{2a}$. 
  The most important formula for polynomials is the Vieta Formulas.

  \begin{formula}
    Vieta Formulas include \\
    Sum of roots ($r_1+r_2+r_3+\cdots+r_n$): $-\frac{a_{n-1}}{a_{n}}$ \\
    Product of roots ($r_1 \cdot r_2 \cdot r_3 \cdots r_n$): $(-1)^n \cdot \frac{a_0}{a_n}$ \\
    Pairwise sums of $p$  ($p=2$: $r_1r_2+r_1r_3+r_1r_4+\cdots+r_{n-1}r_n$): $(-1)^p \cdot \frac{a_{n - p}}{a_{n}}$
  \end{formula}


  \begin{theorem}
    Fundamental theorem of algebra states that a single variable polynomial with degree
    $n$ has exactly $n$ complex roots.
  \end{theorem}

\end{document}