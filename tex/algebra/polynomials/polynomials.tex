\section{Polynomials}
\subsection{Definition and Basics}
A polynomial is defined as $P(x) = a_nx^n + a_{n-1}x^{n-1} + \cdots + a_2x^2 + a_1x + a_0$ 
with names corresponding to their degree (constant, linear, quadratic, cubic, quartic).

The factored form is written as $P(x) = a(x-r)(x-p)\cdots(x-q)$.
The simplest and most useful polynomial is the quadratic. It can be written as $ax^2+bx+c$ and factored respectively.
The formula to solve for $x$ is $x=\frac{-b \pm \sqrt{b^2 - 4ac}}{2a}$. 
The most important formula for polynomials is the Vieta Formulas.

\begin{formula}[Vieta Formulas]
  Sum of roots ($r_1+r_2+r_3+\cdots+r_n$): $-\frac{a_{n-1}}{a_{n}}$ \\
  Product of roots ($r_1  r_2  r_3 \cdots r_n$): $(-1)^n \cdot \frac{a_0}{a_n}$ \\
  Pairwise sums of $p$  ($p=2$: $r_1r_2+r_1r_3+r_1r_4+\cdots+r_{n-1}r_n$): $(-1)^p \cdot \frac{a_{n - p}}{a_{n}}$
\end{formula}

\begin{theorem}[Fundamental theorem of algebra]
  It states that a single variable polynomial with degree
  $n$ has exactly $n$ complex roots.
\end{theorem}

\begin{problem}
  Let $r,s,$ and $t$ be the roots of $3x^3-4x^2+5x+7=0$. (\ia, 8.20 pg.249)
  \begin{enumerate}
    \item Find $r+s+t$ ($\frac{4}{3}$).
    \item Find $r^2+s^2+t^2$ ($\frac{-14}{9}$).
    \item Find $\frac{1}{r} + \frac{1}{s} + \frac{1}{t}$ ($\frac{-5}{7}$).
  \end{enumerate}
\end{problem}

\subsection{Synthetic Division}
A simplification of traditional polynomial division. Note this only works when the coefficients of 
the linear term in the divisor is $1$. It is also know as the \textbf{Ruffini's Rule}.

\textbf{Example:}
$\begin{array}{c|rrrrr} 3&\parbox[b]{0.2in}{\raggedleft 1}&\parbox[b]{0.2in}{\raggedleft -3}&\parbox[b]{0.2in}{\raggedleft 7}&\parbox[b]{0.2in}{\raggedleft -1}&\parbox[b]{0.2in}{\raggedleft 5}\\ &&{3}&{0}&21&60 \\\cline{2-6} \multicolumn{2}{r}{{1}}& {0}&7&20&\multicolumn{1}{|r}{65} \end{array}$

Which is the same as $(x^4 - 3x^3 + 7x^2 - x + 5) \div (x - 3) = x^3+7x+20+\frac{65}{x-3}$. Notice you work from left to right, and multiply to get the next number in the second row.
If your divisor doesn't have $1$ as its coefficient in the linear term, you can divide it by $1/n$ and in the end also multiply the quotient and remainder by $1/n$.

Usually, you write the result of polynomial division as $\frac{f(x)}{d(x)}=q(x)+\frac{r(x)}{d(x)}$.

\subsection{Rational Root Theorem}
\begin{theorem}[Rational Root Theorem]
  A rational root of a polynomial in the form $\pm\frac{p}{q}$ where $p$ and $q$ are relatively prime must follow the condition
  $p | a_0$ and $q | a_n$.
\end{theorem}

\subsection{Remainder Theorem}
\begin{theorem}[Remainder Theorem]
  When a polynomial $f(x)$ is divided by $x-a$, the remainder is determined by $f(a)$.
\end{theorem}

Theorem 1.3 can be proven using the form $\frac{f(x)}{d(x)}=q(x)+\frac{r(x)}{d(x)}$ and synthetic division.

\begin{proof}[Remainder Theorem]
  \begin{align*}
    f(x) &= (x-a)q(x) + r(x) \\
    &= (x-a)q(x)+c. \\
    f(a)&=(a-a)q(a)+c \\
    &=0 \cdot q(a) + c = c
  \end{align*}
  Thus, $f(a)$ always returns the remainder of $f(x) \div (x-a)$.
\end{proof}

\subsection{Factor Theorem}
\begin{theorem}[Factor Theorem]
given the expression $x-a$, it is a divisor of $p(x)$ if and only if $p(a)=0$. This can be proven with the remainder theorem.
\end{theorem}

\subsection{Miscellaneous}
To find the sum of the coefficients of a polynomial $P(x)$, plug $x=1$! 
Example problem: Practice problem \#2
 
\subsection{Practice Problems}
Problems from \ia.
\begin{enumerate}
  \item \sout{6.11, pg. 181}
  \item 6.21 pg. 191
  \item 6.17, pg. 187
  \item 6.22, pg. 191
  \item 6.27, pg. 191
  \item 6.29, pg. 192
  \item $\star$ Challenge Problems, pg. 192
  \item \url{https://numbertheoryguydotcom.files.wordpress.com/2016/03/polynomials.pdf}
\end{enumerate}